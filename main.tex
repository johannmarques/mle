\documentclass{homework}
\author{Johann Marques Viana Freitas}
\class{Econometria I: Prof. Nathalie Gimenes}
\date{\today}
\title{Lista Prática}
\address{}

%\graphicspath{{./media/}}
\usepackage[portuguese]{babel}
\usepackage{amssymb}
\usepackage{amsmath}
\usepackage{bm}
\usepackage{tcolorbox}

\usepackage{Sweave}
\begin{document} \maketitle
\Sconcordance{concordance:main.tex:main.Rnw:%
1 140 1 50 0 1 19 17 1 4 0 23 1 4 0 18 1 8 0 12 1 4 0 10 1 7 0 8 1 4 0 %
10 1 7 0 15 1 22 0 44 1 4 0 13 1 4 0 4 1 8 0 14 1 6 0 4 1 8 0 14 1 4 0 %
4 1 8 0 35 1 4 0 4 1 8 0 27 1 4 0 4 1 8 0 18 1 4 0 4 1 8 0 30 1 9 0 9 1}


\question

\section{Funções de Verossimilhança e Estimadores}

\begin{enumerate}
  \item A mistura de normais é caracterizada da seguinte forma:
  \begin{align}
    e_i \sim \begin{cases}
                \mathrm{N}(\mu_1, \sigma_1^2), \text{com probabilidade } 1/2,\\
                \mathrm{N}(\mu_2, \sigma_2^2), \text{com probabilidade } 1/2.
              \end{cases}
  \end{align}
  
  Para computar a densidade, podemos modelar $f(e_i) \sim \mathrm{Bern}(1/2)$ e calcular $\mathbb{E}_{\mathrm{Bern(1/2)}}\left[f(e_i)\right]$:
  
  \begin{align}
    \mathbb{E}_{\mathrm{Bern(1/2)}}\left[f(e_i)\right] &=
    \dfrac{1}{2}\times\left[\dfrac{1}{\sqrt{2\pi\sigma_1^2}}\mathrm{exp}\left(-\dfrac{1}{2\sigma_1^2}\left(y_i - \beta_0 - \beta_1 x_i - \mu_1 \right)^2\right) +
    \dfrac{1}{\sqrt{2\pi\sigma_2^2}}\mathrm{exp}\left(-\dfrac{1}{2\sigma_2^2}\left(y_i - \beta_0 - \beta_1 x_i - \mu_2 \right)^2\right)\right].
  \end{align}
  
  Podemos escrever a função de verossimilhança amostral como
  
  \begin{align}
    L(\bm{\beta}|\mathbf{y}, \mathbf{x}; \mu_1, \mu_2, \sigma_1^2, \sigma_2^2) &= \prod_{i=1}^n\dfrac{1}{2}\times\left[\dfrac{1}{\sqrt{2\pi\sigma_1^2}}\mathrm{exp}\left(-\dfrac{1}{2\sigma_1^2}\left(y_i - \beta_0 - \beta_1 x_i - \mu_1 \right)^2\right) +\right\\
    &\left\dfrac{1}{\sqrt{2\pi\sigma_2^2}}\mathrm{exp}\left(-\dfrac{1}{2\sigma_2^2}\left(y_i - \beta_0 - \beta_1 x_i - \mu_2 \right)^2\right)\right] \nonumber\\
    &= \dfrac{1}{2^n}\prod_{i=1}^n\times\left[\dfrac{1}{\sqrt{2\pi\sigma_1^2}}\mathrm{exp}\left(-\dfrac{1}{2\sigma_1^2}\left(y_i - \beta_0 - \beta_1 x_i - \mu_1 \right)^2\right) +\right\nonumber\\
    &\left\dfrac{1}{\sqrt{2\pi\sigma_2^2}}\mathrm{exp}\left(-\dfrac{1}{2\sigma_2^2}\left(y_i - \beta_0 - \beta_1 x_i - \mu_2 \right)^2\right)\right]. \nonumber
  \end{align}
  
  Obtendo a log-verossimilhança:
  
  \begin{align}
    l(\bm{\beta}|\mathbf{y}, \mathbf{x}; \mu_1, \mu_2, \sigma_1^2, \sigma_2^2) &= -n\mathrm{log}(2)+\sum^n_{i=1}\mathrm{log}\left(\dfrac{1}{\sqrt{2\pi\sigma_1^2}}\mathrm{exp}\left(-\dfrac{1}{2\sigma_1^2}\left(y_i - \beta_0 - \beta_1 x_i - \mu_1 \right)^2\right) +\right\\
    &\left\dfrac{1}{\sqrt{2\pi\sigma_2^2}}\mathrm{exp}\left(-\dfrac{1}{2\sigma_2^2}\left(y_i - \beta_0 - \beta_1 x_i - \mu_2 \right)^2\right)\right)\nonumber.
  \end{align}
  
  O Estimador de Máxima Verossimilhança é dado por
  
  \begin{align}
    \hat{\bm{\beta}} &= \underset{\beta_0, \beta_1}{\mathrm{argmax}}\ l(\bm{\beta}|\mathbf{y}, \mathbf{x}; \mu_1, \mu_2, \sigma_1^2, \sigma_2^2).
  \end{align}
  
  \item Função de verossimilhança amostral para $u$, com $u_i \sim \Gamma(a,b)$, definida para $u_i > 0$:
  \begin{align}
  L(\bm{\beta}|\mathbf{y}, \mathbf{x}; a, b) &= \prod_{i=1}^n{\dfrac{b^a}{\Gamma(a)}\left(y_i - \beta_1 - \beta_2 x_i\right)^{a-1}\mathrm{exp}\left(-b\left(y_i - \beta_1 - \beta_2 x_i\right)\right)}\\
  &= \left(\dfrac{b^a}{\Gamma(a)}\right)^n\prod_{i=1}^n{\left(y_i - \beta_1 - \beta_2 x_i\right)^{a-1}}\mathrm{exp}\left(-b\sum_{i=1}^n\left(y_i - \beta_1 - \beta_2 x_i\right)\right).\nonumber
  \end{align}
  
  Aplicando tranformação logarítmica
  
  \begin{align}
  l(\bm{\beta}|\mathbf{y}, \mathbf{x}; a, b) &= n\mathrm{log}\left(\dfrac{b^a}{\Gamma(a)}\right) + (a-1)\sum_{i=1}^n{\mathrm{log}\left(y_i - \beta_1 - \beta_2 x_i\right)} -b\sum_{i=1}^n\left(y_i - \beta_1 - \beta_2 x_i\right).
  \end{align}
  
  O Estimador de Máxima Verossimilhança é dado por
  
  \begin{align}
    \hat{\bm{\beta}} &= \underset{\beta_0, \beta_1}{\mathrm{argmax}}\ l(\bm{\beta}|\mathbf{y}, \mathbf{x}; a, b).
  \end{align}
  
  \item Sabemos que $v_i \sim \mathrm{Cauchy}(0,1) \sim t(\nu = 1)$, em que $\nu$ representa os graus de liberdade. Portanto, a função de verossimilhança amostral para $\mathbf{v}$, com $v_i \sim \mathrm{Cauchy}(0,1)$ pode ser escrita como:
  
  \begin{align}
    L(\bm{\beta}|\mathbf{y}, \mathbf{x}; \nu) &= \prod_{i=1}^n\dfrac{\Gamma\left(\dfrac{\nu+1}{2}\right)}{\sqrt{\nu\pi}\Gamma\left(\dfrac{\nu}{2}\right)}\left(1 + \dfrac{\left(y_i - \beta_1 - \beta_2 x_i\right)^2}{\nu}\right)\nonumber\\
    &= \left[\dfrac{\Gamma\left(\dfrac{\nu+1}{2}\right)}{\sqrt{\nu\pi}\Gamma\left(\dfrac{\nu}{2}\right)}\right]^n\prod_{i=1}^n\left(1 + \dfrac{\left(y_i - \beta_1 - \beta_2 x_i\right)^2}{\nu}\right).
  \end{align}
  
  A log-verossimilhança, por sua vez,
  
  \begin{align}
    l(\bm{\beta}|\mathbf{y}, \mathbf{x}; \nu) &= n\left[\mathrm{log}\left(\Gamma\left(\dfrac{\nu+1}{2}\right)\right)-
    \mathrm{log}\left(\sqrt{\nu\pi}\right)-
    \mathrm{log}\left(\Gamma\left(\dfrac{\nu}{2}\right)\right)\right] + \sum_{i=1}^n\mathrm{log}\left(1 + \dfrac{\left(y_i - \beta_1 - \beta_2 x_i\right)^2}{\nu}\right).
  \end{align}
  
  O Estimador de Máxima Verossimilhança é dado por
  
  \begin{align}
    \hat{\bm{\beta}} &= \underset{\beta_0, \beta_1}{\mathrm{argmax}}\ l(\bm{\beta}|\mathbf{y}, \mathbf{x}; \nu).
  \end{align}

\end{enumerate}

\section{Distribuições Assintóticas}

Vide \cite{greene2011econometric}, Teorema 14.1.M2, \textit{se são satisfeitas condições de regularidade},

\begin{align}
  \hat{\bm{\theta}} \xrightarrow{d} \mathrm{N}\left(\bm{\theta}_0, \{\mathbf{I(\bm{\theta}_0)}\}^{-1}),
\end{align}

em que $\hat{\bm{\theta}}$ é o estimador de Máxima Verossimilhança de $\bm{\theta}_0$, e $\mathbf{I(\bm{\theta}_0)} = -\mathbb{E}\left[\dfrac{\partial^2\mathrm{log}(L)}{\partial\bm{\theta}_0\partial\bm{\theta}_0^'}\right].

\begin{tcolorbox}
{\hfill\textbf{Condições de regularidade}\hfill}

blá blá blá

\end{tcolorbox}

% citations
\bibliographystyle{plain}
\bibliography{citations}

\end{document}
